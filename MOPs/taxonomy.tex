\documentclass[autodetect-engine,dvipdfmx-if-dvi,ja=standard,a4paper,12pt]{bxjsarticle}
    \usepackage{fancyhdr}
    \usepackage{titlesec}
    \usepackage{graphicx}
    \usepackage{multicol}
    
    \setlength{\textheight}{\paperheight}
    \setlength{\topmargin}{-10truemm}
    \addtolength{\topmargin}{-\headheight}
    \addtolength{\topmargin}{-\headsep}
    \addtolength{\textheight}{-30truemm}
    
    \setlength{\textwidth}{\paperwidth}
    \setlength{\oddsidemargin}{-10.0truemm}
    \setlength{\evensidemargin}{-10.0truemm}
    \addtolength{\textwidth}{-30truemm}
    
    \titleformat*{\section}{\large\bfseries}
    \titleformat*{\subsection}{\leftskip=1em\normalsize\bfseries}
    
    \pagestyle{fancy}
    \lhead{Satoshi Koda}
    \rhead{染谷研究室 文献解説資料}
    \renewcommand{\headrulewidth}{0pt}
    
    \begin{document}
      \begin{center}
      \large{パレートフロントを可視化する手法の分類} \\
      A Taxonomy of Methods for Visualizing Pareto Front Approximations\\
      \normalsize Bogdan Filipic, Tea Tusar\\
      \vspace*{10pt}
      \begin{abstract}
      多目的最適化では,従来の汎用的なデータの可視化手法から,多目的最適化の特殊性(専門性?)に合わせたアプローチに至るまで,
      多くの技術を利用して結果の可視化が行われている.また近年,専門的なアプローチの数が急速に増加している.\\
      そこで,我々はこの分野のユーザと開発者の両方を支援するために,パレートフロント近似を可視化するための手法の分類法を提案する.\\
      これは視覚的表現が使用されたものより,視覚化されたデータの性質と視覚化手法の特性に基づいており,一度の実行により得られた個体の視覚化および複数回の試行による複数の近似セットをカバーする.視覚化手法の例を用いて提案する分類カテゴリを特徴づけ,図示する.\\
      本提案により,多目的最適化のコミュニティの本質を突き,コミュニケーションを容易にし,視覚化手法のさらなる開発のために役立つことが期待される.
      
      \vspace*{-10pt}
      \section*{キーワード}
      \vspace{-10pt}
      多目的最適化, 可視化手法, 分類体系
      \end{abstract}
      \end{center}

    \begin{multicols}{2}
%    \begin{flushleft}
      \section{はじめに}
      多目的最適化では,複数の競合する目的を持つ最適化問題の解の探索を扱う.
      理想的な多目的最適化アルゴリズムは,Approximation setとしてパレートフロントを近似する非劣解集合が得られ,
      クオリティを評価し,Approximation setを比較すると同時に,可視化により結果を精査する補完的な手法として用いることができる.
      可視化は,多目的最適化に関する多くのタスク,例えばパレートフロントの特性の推定,目的間のトレードオフの調査,最適解の特定,アルゴリズムの収束性の追跡,アルゴリズムのパフォーマンスの比較,対話的な最適化の手助けなどに有用である.\\
      データの可視化の広い分野では,従来可視化されるデータの特性に応じて分類され,科学的視覚化[22]と情報視覚化[37]の区別が採用されている.
      この分離は実用的であるが,2つの小分野の橋渡しを妨げる可能性がある.それらがどのように関連しているか[39]を理解するために,手法の特性を考慮し,ユーザとその概念モデルを含む可視化手法の代替的な分類法を紹介します.\\
      一方,多目的最適化を考慮して可視化手法を分類する試みはあまり多くは報告されていない.Pareto front approximationsを可視化した手法[42]から,多目的最適化の分野以外で設計された一般的な多次元データ可視化手法と,多目的最適化における可視化のために特別に設計された手法を区別した.
      前者は解集合のパレート優越関係や多目的最適化に特有の特徴を維持する工夫を行っていないが,後者はこれらの制約を満たす試みとして設計された.\\
      {[4]}では,多目的最適化における知見を得るために可視化技術が調査されている.著者らは,最適化プロセスで生成された解から問題に関する知識を抽出し,その方法論と種類によってそれらを分類するために利用可能な,多数のデータマイニング手法を挙げている.
      彼らのトップレベルの分類は,記述統計,ビジュアルデータマイニング,機械学習の手法で構成されているが,ビジュアルデータマイニングの手法は,グラフィカルビジュアライゼーション,クラスタリングに基づくビジュアライゼーション,多様な学習方法に分類される.
      私達の知る限り,これは多目的最適化のための可視化手法の最も詳細な分類である.\\
      多目的最適化の進展に伴い,可視化手法の数も急速に増加している.

      \section{提案-分類方法-}
      前者のカテゴリ(Single approximation sets)は、一度に1つまたはいくつかの近似集合を主に表す方法を表す。複数のセットが表示されても、1つ以上のアルゴリズムの結果に関係なく、個別に処理されます。 これらの方法は、そのような個々のプロット(アニメーション)のシーケンスを生成することによってアルゴリズムの進行を監視するためにも使用することができる。\\
      後者(Repeated approximation sets)は,複数のアルゴリズム実行または最適化プロセスの進化全体を1つのプロットで表すことを目的としています。これは、客観的な空間の各点でオプティマイザのパフォーマンスに関する情報を収集し、この関数を視覚化する関数を導入することによって実現されます。



    \end{multicols}   
%    \end{flushleft}
    \end{document}
